\documentclass[a4paper,11pt]{article}

\usepackage[utf8]{inputenc}
\usepackage[T1]{fontenc}
\usepackage[ngerman]{babel}
\usepackage{graphicx}
\usepackage{array}
\usepackage{booktabs}
\usepackage{amsmath}
\usepackage{float}

\usepackage[scaled]{helvet}
\renewcommand{\familydefault}{\sfdefault}

\usepackage[a4paper, top=2cm, bottom=2cm, left=2cm, right=2cm]{geometry}

\usepackage{setspace}
\setstretch{1.5}

\setlength{\parindent}{0pt}
\setlength{\parskip}{6pt}

\usepackage{microtype}
\sloppy
\hyphenpenalty=1000
\tolerance=3000

\renewcommand{\footnotesize}{\fontsize{10}{12}\selectfont}

\setcounter{secnumdepth}{3}
\setcounter{tocdepth}{3}

\usepackage{titlesec}
\titleformat{\section}{\normalfont\fontsize{12}{14}\bfseries}{\thesection}{1em}{}
\titleformat{\subsection}{\normalfont\fontsize{12}{14}\bfseries}{\thesubsection}{1em}{}
\titleformat{\subsubsection}{\normalfont\fontsize{12}{14}\bfseries}{\thesubsubsection}{1em}{}

\usepackage[
  colorlinks=true,
  linkcolor=black,
  citecolor=blue,
  filecolor=black,
  urlcolor=blue
]{hyperref}
\usepackage[capitalise,nameinlink]{cleveref}

\usepackage{fancyhdr}
\pagestyle{fancy}
\fancyhf{}
\renewcommand{\headrulewidth}{0pt}
\fancyfoot[C]{\thepage}

\usepackage[backend=biber, style=apa]{biblatex}
\addbibresource{references.bib}

\usepackage{titling}

\usepackage{acronym}
\usepackage[german=quotes]{csquotes}

\usepackage{caption}
\usepackage{threeparttable}
\captionsetup[table]{
    font=small,
    skip=10pt,
    labelfont=bf
}

\usepackage{listings}
\usepackage{xcolor}

\definecolor{codegreen}{rgb}{0,0.6,0}
\definecolor{codegray}{rgb}{0.5,0.5,0.5}
\definecolor{codepurple}{rgb}{0.58,0,0.82}
\definecolor{backcolour}{rgb}{0.95,0.95,0.92}

\lstdefinestyle{mystyle}{
    backgroundcolor=\color{backcolour},
    commentstyle=\color{codegreen},
    keywordstyle=\color{magenta},
    numberstyle=\tiny\color{codegray},
    stringstyle=\color{codepurple},
    basicstyle=\ttfamily\footnotesize,
    breakatwhitespace=false,
    breaklines=true,
    captionpos=b,
    keepspaces=true,
    numbers=left,
    numbersep=5pt,
    showspaces=false,
    showstringspaces=false,
    showtabs=false,
    tabsize=2
}
\lstset{style=mystyle}

\begin{document}

\begin{titlepage}
    \thispagestyle{empty}
    \centering
    \vspace*{5cm}
    {\Huge\bfseries Projekt:
Mobile Robotik DLBROESR01\_D \par}
    \vspace{1cm}
    {\Large Fallstudie \par}
    \vspace{0.5cm}
    {\large Studiengang: Angewandte Künstliche Intelligenz \par}
    \vspace{0.5cm}
    {\large Sven Behrens \par}
    \vspace{0.5cm}
    {\large Matrikelnummer: 42303511 \par}
    \vspace{0.5cm}
    {\large Prof. Dr. Florian Simroth \par}
    \vspace{0.5cm}
    {\large \today \par}
\end{titlepage}

\pagenumbering{Roman}
\setcounter{page}{1}

\tableofcontents
\newpage

\listoffigures
\addcontentsline{toc}{section}{Abbildungsverzeichnis}
\newpage

\listoftables
\addcontentsline{toc}{section}{Tabellenverzeichnis}
\newpage

\section*{Abkürzungsverzeichnis}
\addcontentsline{toc}{section}{Abkürzungsverzeichnis}
\begin{acronym}[C-Space]
    \acro{AMR}{Automated Mobile Robot}
    \acro{C-Space}{Configuration Space}
    \acro{PRM}{Probabilistic Roadmap}
    \acro{RRT}{Rapidly-exploring Random Tree}
\end{acronym}
\newpage

\pagenumbering{arabic}
\setcounter{page}{1}

\section{Einleitung}
Die Intralogistik befindet sich in einem tiefgreifenden Wandel. Angetrieben durch die steigende
Kundenerwartungen an Liefergeschwindigkeit setzen immer mehr Unternehmen auf automatisierte
mobile Roboter (\ac{AMR}) für den innerbetrieblichen Warentransport \parencite{fragapane2021amr}. 
Eine zentrale Herausforderung beim Einsatz solcher Systeme stellt die Pfadplanung dar: Der Roboter muss 
in der Lage sein, kollisionsfrei von einem Startpunkt zu einem Zielpunkt zu navigieren und dabei sowohl 
statische Hindernisse als auch die eigene Geometrie zu berücksichtigen. Vor diesem Hintergrund wurde im
Rahmen des Moduls \enquote{Mobile Robotik} an der IU Internationalen Hochschule ein Softwareprototyp
entwickelt, der verschiedene etablierte Pfadplanungsalgorithmen implementiert und vergleichend evaluiert.

Das primäre Projektziel bestand in der Implementierung eines Softwaresystems, das für einen omnidirektionalen 
Roboter in einer polygonbasierten Lagerumgebung kollisionsfreie Pfade berechnet. Die zentrale Forschungsfrage 
konzentrierte sich darauf, wie verschiedene Pfadplanungsalgorithmen hinsichtlich Pfadqualität, Rechenzeit und 
Robustheit in unterschiedlich komplexen Umgebungen performieren. Besondere Aufmerksamkeit galt dabei der 
korrekten Konstruktion des Konfigurationsraums (\ac{C-Space}), der die Robotergeometrie in die Hindernisdarstellung 
integriert.

Die methodische Vorgehensweise gliederte sich in mehrere aufeinander aufbauende Phasen. Zunächst wurde eine 
modulare Softwarearchitektur in Python entwickelt, die eine klare Trennung zwischen Umgebungsmodellierung, 
Robotergeometrie, Konfigurationsraumberechnung und Pfadplanung vorsieht. Anschließend wurden fünf verschiedene 
Algorithmen implementiert und verglichen: Die graphbasierten Verfahren A*, Dijkstra und Best-First-Search sowie 
die samplingbasierten Methoden \ac{RRT} und \ac{PRM}. Die Evaluation erfolgte auf drei Testumgebungen mit 
steigender Komplexität unter Verwendung von drei unterschiedlichen Robotergeometrien.

Der gewählte Ansatz zeichnet sich durch seine Erweiterbarkeit aus. Neben dem zweidimensionalen \ac{C-Space} für 
omnidirektionale Roboter wurde zusätzlich ein dreidimensionaler Konfigurationsraum implementiert, der die Orientierung 
des Roboters als dritte Dimension berücksichtigt. Dies ermöglicht die Pfadplanung für nicht-holonome Roboter 
und demonstriert die Skalierbarkeit des entwickelten Systems. Durch die systematische Evaluation verschiedener 
Algorithmus-Roboter-Umgebungs-Kombinationen wurden quantitative Erkenntnisse gewonnen, die als Entscheidungsgrundlage 
für den praktischen Einsatz dienen können.

Die vorliegende Fallstudie gliedert sich wie folgt: Nach der Beschreibung der Projektumgebung wird der iterative 
Entwicklungsansatz erläutert. Die Hauptabschnitte behandeln die 2D- und 3D-Pfadplanung mit ihren jeweiligen Ergebnissen. 
Abschließend werden die Projektergebnisse kritisch reflektiert und Verbesserungspotenziale aufgezeigt.

\section{Hauptteil}
\subsection{Projektumgebung}
\subsection{Iterativer Analyseansatz}
\subsection{2D}
\subsection{3D}

\section{Fazit}
\subsection{Zielerreichung und Projektergebnisse}
\subsection{Kritische Reflexion}
\subsection{Verbesserungspotenziale und Optimierungsansätze}
\subsection{Ausblick}

\section*{Projektrepository}
\addcontentsline{toc}{section}{Projektrepository}
Der vollständige Quellcode ist im GitHub-Repository verfügbar: \url{https://github.com/svenb23/mobile-robotik-pfadplanung}

\newpage

\printbibliography
\addcontentsline{toc}{section}{Literaturverzeichnis}

\newpage
\section*{Verzeichnis der Anhänge}
\addcontentsline{toc}{section}{Verzeichnis der Anhänge}

\appendix
\section*{Anhang}
\addcontentsline{toc}{section}{Anhang}

\end{document}
